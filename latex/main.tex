\documentclass[a4paper]{article}
\usepackage{template}
\usepackage{array}
\begin{document}

\begin{titlepage}
\begin{center}
ĐẠI HỌC QUỐC GIA THÀNH PHỐ HỒ CHÍ MINH \\
TRƯỜNG ĐẠI HỌC BÁCH KHOA  \\
KHOA KHOA HỌC VÀ KĨ THUẬT MÁY TÍNH
\end{center}

\vspace{1cm}

\begin{figure}[h!]
\begin{center}
\includegraphics[width=3cm]{images/hcmut.png}
\end{center}
\end{figure}

\vspace{1cm}

\begin{center}
\rule{\textwidth}{0.4pt} % gạch ngang trên, dày 0.4pt

\vspace{0.3cm}
{\Large\bfseries DATA MINING (CO3029)}\\[0.3cm]
{\Large\bfseries Bài tập lớn - Học kì 251}\\[0.3cm]
{\LARGE\bfseries\itshape DỰ ĐOÁN KHẢ NĂNG KHÁCH HÀNG ĐĂNG KÝ
GỬI TIỀN CÓ KỲ HẠN}\\[0.3cm]

\rule{\textwidth}{0.4pt} % gạch ngang dưới
\end{center}


\vspace{1cm}

\begin{table}[h]
\begin{tabular}{rrl}
\hspace{3.5cm} & \textbf{Advisor(s):} & Thầy Đỗ Thanh Thái\\
& & \\
& & \\
& \textbf{Student(s):} & Trần Đức Trí Cường - 2210443\\
&& Phạm Ngọc Long - 2211894 \\
& & Bùi Trọng Văn - 2213915
\end{tabular}
\end{table}
\vspace{2cm}
\begin{center}
{\footnotesize THÀNH PHỐ HỒ CHÍ MINH, THÁNG 11 - 2025}
\end{center}
\end{titlepage}


%\thispagestyle{empty}

\newpage
\tableofcontents
\newpage


%%%%%%%%%%%%%%%%%%%%%%%%%%%%%%%%%
\section*{Lời nói đầu} 
Nhóm xin chân thành cảm ơn thầy Đỗ Thanh Thái đã phụ trách giảng dạy, cung cấp một số kiến thức nền tảng làm bước đệm để nhóm hoàn thành bài tập lớn học kỳ 251.

Quá trình làm bài tập lớn cũng là quá trình tự học đối với từng thành viên của nhóm. Nhóm đã trau dồi được các kiến thức về khai phá dữ liệu, học máy, giải thuật, kĩ năng làm việc nhóm,... Bên cạnh đó, nhóm còn có cơ hội rèn luyện cách trình bày, báo cáo kết quả một cách khoa học và logic.

Dù gặp không ít khó khăn trong quá trình thực hiện, nhưng nhờ vào sự giúp đỡ, hỗ trợ lẫn nhau, nhóm đã có thể vượt qua và hoàn thành bài tập lớn một cách tốt nhất có thể. Đây là một trải nghiệm quý báu giúp nhóm không chỉ củng cố kiến thức chuyên môn mà còn phát triển thêm nhiều kỹ năng quan trọng cho công việc và học tập sau này.

Nhóm xin gửi lời cảm ơn chân thành đến thầy và hy vọng sẽ tiếp tục nhận được sự hướng dẫn, chia sẻ kiến thức từ thầy trong những bài tập lớn tiếp theo.
\addcontentsline{toc}{section}{Lời nói đầu}

\section{Danh sách thành viên \& Phân chia công việc}
\vspace{0.2cm}
\begin{center}
\begin{tabular}{|c|c|c|c|c|}
\hline
\textbf{STT} & \textbf{Họ và tên} & \textbf{MSSV} & \textbf{Nội dung}                                                                                                               & \textbf{Mức độ hoàn thành} \\ \hline
1           & Trần Đức Trí Cường       & 2210443       & \begin{tabular}[c]{@{}c@{}}Làm model, soạn report, slide \end{tabular} & 100\%                         \\ \hline
2           & Phạm Ngọc Long      &  2211894      & \begin{tabular}[c]{@{}c@{}}Xử lí, so sánh, nhận xét dữ liệu, soạn report \end{tabular}                            & 100\%                         \\ \hline
3           & Bùi Trọng Văn      & 2213915       & \begin{tabular}[c]{@{}c@{}}Xử lí, so sánh, nhận xét dữ liệu, soạn report \end{tabular}                            & 100\%                         \\ \hline
\end{tabular}
\end{center}
\newpage

%%%%%%%%%%%%%%%%%CONTENTS%%%%%%%%%%%%%%%%%%%
\section{Giới thiệu}

\subsection{Bối cảnh nghiên cứu}
\hspace{0.5cm} Trong bối cảnh ngân hàng bán lẻ cạnh tranh ngày càng gay gắt, các tổ chức tài chính phải liên tục triển khai các chiến dịch marketing trực tiếp (đặc biệt là telemarketing qua điện thoại) để giới thiệu sản phẩm tiền gửi có kỳ hạn cho khách hàng hiện hữu và khách hàng tiềm năng. Tuy nhiên, việc gọi điện “đại trà” cho toàn bộ danh sách khách hàng vừa tốn kém chi phí nhân sự, vừa gây phiền hà cho những khách hàng không có nhu cầu thực sự. Do đó, nhu cầu phân tích, xử lý dữ liệu, ứng dụng các mô hình học máy để dự đoán trước nhóm khách hàng có khả năng đăng ký tiền gửi có kỳ hạn là rất cấp thiết. 

\subsection{Mục tiêu nghiên cứu}
\hspace{0.5cm}Dự án này tập trung vào việc phân tích, xử lý dữ liệu, sử dụng các mô hình học máy để dự đoán biến mục tiêu \texttt{y} (khách hàng có đăng ký tiền gửi có kỳ hạn hay không, với hai giá trị \texttt{yes} và \texttt{no}) dựa trên tập dữ liệu \texttt{train.csv} của cuộc thi \textit{Marketing Dataset} trên Kaggle. Cụ thể, các mục tiêu chính bao gồm:

\begin{itemize}
    \item Phân tích, so sánh, khảo sát và tiền xử lý dữ liệu khách hàng từ tập \texttt{train.csv} (xử lý giá trị thiếu, mã hoá biến phân loại, chuẩn hoá/chuẩn chỉnh dữ liệu nếu cần)
    \item Xây dựng và so sánh nhiều mô hình(Logistic Regression, KNN, SVM, Decision Tree, Random Forest, \dots)
    \item Đánh giá hiệu suất của các mô hình dự đoán
\end{itemize}

\subsection{Bộ dữ liệu}
\hspace{0.5cm} \textbf{Link Marketing dataset:} \url{https://www.kaggle.com/competitions/marketing-data/overview}

\hspace{0.5cm} Bộ dữ liệu được sử dụng trong dự án này xuất phát từ các chiến dịch marketing trực tiếp qua điện thoại của một ngân hàng tại Bồ Đào Nha . Dữ liệu bao gồm:

\begin{itemize}
    % \item Thời gian thu thập: Trong giai đoạn 2008–2010
    \item Số lượng mẫu: 3,000 mẫu
    \item Số lượng biến: 21 biến (20 biến đầu vào và 1 biến mục tiêu \texttt{y})
    \item \textbf{Biến mục tiêu:}
    \begin{itemize}
        \item \texttt{y}: Khách hàng có đăng ký tiền gửi có kỳ hạn hay không (\texttt{yes} / \texttt{no}) ?
    \end{itemize}
    \item \textbf{Nhóm biến đầu vào chính:}
    \begin{itemize}
        % \item Thông tin cá nhân của khách hàng:
        % \begin{itemize}
            \item \texttt{age}: tuổi khách hàng (số)
            \item \texttt{job}: nghề nghiệp (admin., blue-collar, services, \dots)
            \item \texttt{marital}: tình trạng hôn nhân (single, married, divorced)
            \item \texttt{education}: trình độ học vấn (basic, high.school, university.degree, \dots)
            \item \texttt{default}: có nợ xấu hay không (yes/no)
            \item \texttt{housing}: có vay mua nhà hay không (yes/no)
            \item \texttt{loan}: có vay tiêu dùng hay không (yes/no)
        % \end{itemize}
        
        % \item Thông tin về lần liên hệ hiện tại trong chiến dịch:
        % \begin{itemize}
            \item \texttt{contact}: kênh liên hệ (cellular, telephone)
            \item \texttt{month}: tháng liên hệ cuối cùng (jan, feb, \dots, dec)
            \item \texttt{day\_of\_week}: ngày trong tuần liên hệ cuối cùng (mon, tue, \dots, fri)
            \item \texttt{duration}: thời lượng cuộc gọi cuối cùng (giây)
        % \end{itemize}
        
        % \item Thông tin về lịch sử và cường độ chiến dịch:
        % \begin{itemize}
            \item \texttt{campaign}: số lần liên hệ trong chiến dịch hiện tại với khách hàng này
            \item \texttt{pdays}: số ngày kể từ lần liên hệ trước đó (999 nếu chưa từng liên hệ)
            \item \texttt{previous}: số lần liên hệ trước chiến dịch hiện tại
            \item \texttt{poutcome}: kết quả của chiến dịch marketing trước đó (success, failure, nonexistent)
        % \end{itemize}
        
        % \item Các biến vĩ mô, bối cảnh kinh tế - xã hội:
        % \begin{itemize}
            \item \texttt{emp.var.rate}: tỷ lệ biến động việc làm (chỉ báo hàng quý)
            \item \texttt{cons.price.idx}: chỉ số giá tiêu dùng (hàng tháng)
            \item \texttt{cons.conf.idx}: chỉ số niềm tin người tiêu dùng (hàng tháng)
            \item \texttt{euribor3m}: lãi suất Euribor 3 tháng (hàng ngày)
            \item \texttt{nr.employed}: số lượng lao động (chỉ báo hàng quý)
        % \end{itemize}
    \end{itemize}
    
    % \item \textbf{Đặc điểm phân phối mục tiêu:}
    % \begin{itemize}
    %     \item Biến \texttt{y} mất cân bằng mạnh: phần lớn quan sát mang nhãn \texttt{``no''}, số lượng \texttt{``yes''} chiếm tỷ lệ nhỏ, điều này đặt ra bài toán xử lý dữ liệu mất cân bằng trong quá trình huấn luyện mô hình.
    % \end{itemize}
\end{itemize}



\subsection{Ý nghĩa thực tiễn}
\hspace{0.5cm}Nghiên cứu và mô hình hoá bộ dữ liệu marketing ngân hàng này có ý nghĩa thực tiễn quan trọng:

\begin{itemize}
    \item Giúp ngân hàng nhận diện tốt hơn nhóm khách hàng tiềm năng có khả năng đăng ký tiền gửi, từ đó tối ưu hoá danh sách gọi điện trong các chiến dịch telemarketing
    \item Góp phần giảm chi phí chiến dịch (ít cuộc gọi lãng phí hơn), tăng tỷ lệ chuyển đổi, nâng cao hiệu quả sử dụng nguồn lực nhân viên
    \item Cung cấp cơ sở dữ liệu định lượng để thiết kế chiến lược marketing cá nhân hoá, theo từng phân khúc khách hàng khác nhau
\end{itemize}

\subsection{Cấu trúc báo cáo}
\hspace{0.5cm}Báo cáo được tổ chức thành các phần chính sau:

\begin{itemize}
    \item Phần 1: Giới thiệu đề tài và bài toán dự đoán
    \item Phần 2: Khảo sát và tiền xử lý dữ liệu
    \item Phần 3: Xây dựng các mô hình học máy
    \item Phần 4: Đánh giá và so sánh mô hình
    \item Phần 5: Kết luận và hướng phát triển
\end{itemize}

\newpage
\section{Khảo sát và tiền xử lý dữ liệu}


\newpage
\section{Xây dựng các mô hình học máy}
Nhóm sử dụng 9 mô hình học máy sau để dự đoán cho bài toán đã giới thiệu ở trên:
\begin{itemize}
    \item KNN
    \item Logistic Regression
    \item SVM
    \item Decision Tree
    \item Random Forest
    \item XGBoost
    \item Gradient Boosting
    \item Naivebayes
    \item MLP
\end{itemize}
Các mô hình được đánh giá bởi các thông số sau:
\begin{itemize}
    \item Accuracy
    \item Precision
    \item Recall
    \item F1-score
    \item Confussion Matrix
\end{itemize}
\begin{figure}[H]
    \centering
    \includegraphics[width=0.9\linewidth]{images/ML.jpg}
    \label{fig:enter-label}
\end{figure}

\newpage
\subsection{Mô hình KNN}
\subsubsection{Giới thiệu}
\hspace{0.5cm} Mô hình K--Nearest Neighbors (KNN) là một thuật toán phân loại dựa trên ``láng giềng gần nhất''.
Ý tưởng chính là, để dự đoán nhãn của một mẫu mới, mô hình sẽ tìm $k$ điểm dữ liệu gần nhất trong
tập huấn luyện (theo một khoảng cách, thường dùng Euclid) và gán nhãn theo đa số phiếu của các láng giềng này.
KNN là mô hình phi tham số, đơn giản, dễ cài đặt, nhưng có thể tốn thời gian khi dự đoán nếu số lượng mẫu lớn.
\subsubsection{Kết quả huấn luyện mô hình}
\hspace{0.5cm} Configuration: \{'n\_neighbors': 15, 'weights': 'distance', 'algorithm': 'auto', 'leaf\_size': 30, 'p': 1, 'metric': 'manhattan'\}

\hspace{0.5cm} Kết quả trên tập test:
\begin{itemize}
    \item Accuracy : 0.7683
    \item Precision: 0.8441
    \item Recall   : 0.7683
    \item F1-score : 0.7990
\end{itemize}

\begin{table}[H]
    \centering
    \begin{tabular}{lcccc}
        \hline
        & Precision & Recall & F1-score & Support \\
        \hline
        0       & 0.92 & 0.81 & 0.86 & 534 \\
        1       & 0.22 & 0.44 & 0.29 & 66  \\
        \hline
        Accuracy      &      &      & 0.77 & 600 \\
        Macro avg     & 0.57 & 0.62 & 0.58 & 600 \\
        Weighted avg  & 0.84 & 0.77 & 0.80 & 600 \\
        \hline
    \end{tabular}
    \caption{Detailed metrics per class}
\end{table}

\begin{figure}[H]
    \centering
    \begin{minipage}[b]{0.48\linewidth}
        \centering
        \includegraphics[width=\linewidth]{images/KNN_1.png}
        \label{fig:label1}
    \end{minipage}
    \hspace{0.02\linewidth} % khoảng cách giữa 2 hình
    \begin{minipage}[b]{0.48\linewidth}
        \centering
        \includegraphics[width=\linewidth]{images/KNN_2.png}
        \label{fig:label2}
    \end{minipage}
    \caption{Confusion matrix of KNN}
    \label{fig:combined}
\end{figure}

\newpage
\subsection{Mô hình Logistic Regression}
\subsubsection{Giới thiệu}
\hspace{0.5cm} Logistic Regression là một mô hình tuyến tính dùng cho bài toán phân loại nhị phân. Mô hình giả định rằng logit (log odds) của xác suất thuộc lớp dương là một hàm tuyến tính của các biến đầu vào.
Đầu ra của mô hình là xác suất thông qua hàm sigmoid, từ đó ta chọn ngưỡng (thường là 0.5) để phân loại. Ưu điểm của Logistic Regression là dễ huấn luyện, dễ diễn giải hệ số và thường cho kết quả ổn định trên nhiều bộ dữ liệu.
\subsubsection{Kết quả huấn luyện mô hình}
\hspace{0.5cm} Configuration: \{'penalty': 'l2', 'dual': False, 'tol': 0.0001, 'C': 15.0, 'solver': 'saga', 'max\_iter': 5000, 'class\_weight': 'balanced'\}

\hspace{0.5cm} Kết quả trên tập test:
\begin{itemize}
    \item Accuracy : 0.7533
    \item Precision: 0.8495
    \item Recall   : 0.7533
    \item F1-score : 0.7903
\end{itemize}

\begin{table}[H]
    \centering
    \begin{tabular}{lcccc}
        \hline
        & Precision & Recall & F1-score & Support \\
        \hline
        0     &  0.93   & 0.78  &  0.85   &   534 \\
        1       & 0.22  &   0.50  &   0.31     &   66  \\
        \hline
        Accuracy      &      &      & 0.75 & 600 \\
        Macro avg     & 0.57 & 0.64 & 0.58 & 600 \\
        Weighted avg  & 0.85 & 0.75 & 0.79 & 600 \\
        \hline
    \end{tabular}
    \caption{Detailed metrics per class}
\end{table}

\begin{figure}[H]
    \centering
    \begin{minipage}[b]{0.48\linewidth}
        \centering
        \includegraphics[width=\linewidth]{images/log_1.png}
        \label{fig:label1}
    \end{minipage}
    \hspace{0.02\linewidth} % khoảng cách giữa 2 hình
    \begin{minipage}[b]{0.48\linewidth}
        \centering
        \includegraphics[width=\linewidth]{images/log_2.png}
        \label{fig:label2}
    \end{minipage}
    \caption{Confusion matrix of Logistic Regression}
    \label{fig:combined}
\end{figure}

\subsection{Mô hình SVM}
\subsubsection{Giới thiệu}
\hspace{0.5cm} Support Vector Machine (SVM) là mô hình phân loại tìm một siêu phẳng (hyperplane) có khoảng cách biên (margin)
lớn nhất để tách các lớp dữ liệu. Với hạt nhân (kernel), SVM có thể ánh xạ dữ liệu sang không gian đặc trưng
chiều cao hơn để xử lý những bài toán không tuyến tính. SVM thường hoạt động tốt trên dữ liệu có số chiều vừa phải
và phân tách tương đối rõ ràng giữa các lớp.
\subsubsection{Kết quả huấn luyện mô hình}
\hspace{0.5cm} Configuration: \{'C': 15.0, 'kernel': 'rbf', 'degree': 3, 'gamma': 'scale', 'coef0': 0.0, 'class\_weight': 'balanced', 'cache\_size': 1000\}

\hspace{0.5cm} Kết quả trên tập test:
\begin{itemize}
    \item Accuracy : 0.8083
    \item Precision: 0.8260
    \item Recall   : 0.8083
    \item F1-score : 0.8167
\end{itemize}

\begin{table}[H]
    \centering
    \begin{tabular}{lcccc}
        \hline
        & Precision & Recall & F1-score & Support \\
        \hline
        0     & 0.90    &  0.88   &   0.89   &   534 \\
        1       & 0.20  &    0.24   &   0.22     &   66  \\
        \hline
        Accuracy      &      &      & 0.81 & 600 \\
        Macro avg     & 0.55 & 0.56 & 0.55 & 600 \\
        Weighted avg  & 0.83 & 0.81 & 0.82 & 600 \\
        \hline
    \end{tabular}
    \caption{Detailed metrics per class}
\end{table}

\begin{figure}[H]
    \centering
    \begin{minipage}[b]{0.48\linewidth}
        \centering
        \includegraphics[width=\linewidth]{images/svm_1.png}
        \label{fig:label1}
    \end{minipage}
    \hspace{0.02\linewidth} % khoảng cách giữa 2 hình
    \begin{minipage}[b]{0.48\linewidth}
        \centering
        \includegraphics[width=\linewidth]{images/svm_2.png}
        \label{fig:label2}
    \end{minipage}
    \caption{Confusion matrix of SVM}
    \label{fig:combined}
\end{figure}

\newpage
\subsection{Mô hình Decision Tree}
\subsubsection{Giới thiệu}
\hspace{0.5cm} Decision Tree là mô hình dự đoán dựa trên cấu trúc cây, trong đó mỗi nút trong cây là một điều kiện tách dữ liệu
trên một thuộc tính, và mỗi lá cây tương ứng với một nhãn dự đoán. Mô hình được xây dựng bằng cách chọn những phép
tách làm giảm độ hỗn loạn (entropy, Gini) nhiều nhất. Decision Tree dễ trực quan hoá và giải thích, nhưng dễ bị
quá khớp (overfitting) nếu không được cắt tỉa (pruning) hoặc giới hạn độ sâu.
\subsubsection{Kết quả huấn luyện mô hình}
\hspace{0.5cm} Configuration: \{'criterion': 'gini', 'splitter': 'best', 'max\_depth': 35, 'min\_samples\_split': 2, 'min\_samples\_leaf': 1, 'max\_features': 'sqrt', 'class\_weight': 'balanced'\}

\hspace{0.5cm} Kết quả trên tập test:
\begin{itemize}
    \item Accuracy : 0.7950
    \item Precision: 0.8376
    \item Recall   : 0.7950
    \item F1-score : 0.8138
\end{itemize}

\begin{table}[H]
    \centering
    \begin{tabular}{lcccc}
        \hline
        & Precision & Recall & F1-score & Support \\
        \hline
        0     & 0.91   &   0.85   &   0.88   &   534 \\
        1       & 0.22   &   0.35  &    0.27     &   66  \\
        \hline
        Accuracy      &      &      & 0.80 & 600 \\
        Macro avg     & 0.57 & 0.60 & 0.58 & 600 \\
        Weighted avg  & 0.84 & 0.80 & 0.81 & 600 \\
        \hline
    \end{tabular}
    \caption{Detailed metrics per class}
\end{table}

\begin{figure}[H]
    \centering
    \begin{minipage}[b]{0.48\linewidth}
        \centering
        \includegraphics[width=\linewidth]{images/tree_1.png}
        \label{fig:label1}
    \end{minipage}
    \hspace{0.02\linewidth} % khoảng cách giữa 2 hình
    \begin{minipage}[b]{0.48\linewidth}
        \centering
        \includegraphics[width=\linewidth]{images/tree_2.png}
        \label{fig:label2}
    \end{minipage}
    \caption{Confusion matrix of Decision Tree}
    \label{fig:combined}
\end{figure}

\newpage
\subsection{Mô hình Random Forest}
\subsubsection{Giới thiệu}
\hspace{0.5cm} Random Forest là một mô hình tập hợp (ensemble) của nhiều cây quyết định. Mỗi cây được huấn luyện trên một
mẫu bootstrap của dữ liệu và tại mỗi nút chỉ xem xét một tập con ngẫu nhiên các thuộc tính khi tách.
Dự đoán cuối cùng được lấy bằng cách bỏ phiếu đa số từ các cây thành phần. Nhờ cơ chế lấy trung bình qua nhiều cây,
Random Forest thường giảm được hiện tượng overfitting và cho hiệu suất tốt, ổn định.
\subsubsection{Kết quả huấn luyện mô hình}
\hspace{0.5cm} Configuration: \{'n\_estimators': 1500, 'criterion': 'gini', 'max\_depth': 20, 'min\_samples\_split': 2, 'min\_samples\_leaf': 1, 'max\_features': 'sqrt', 'bootstrap': True, 'max\_samples': 0.9, 'class\_weight': 'balanced\_subsample'\}

\hspace{0.5cm} Kết quả trên tập test:
\begin{itemize}
    \item Accuracy : 0.8817
    \item Precision: 0.8602
    \item Recall   : 0.8817
    \item F1-score : 0.8679
\end{itemize}

\begin{table}[H]
    \centering
    \begin{tabular}{lcccc}
        \hline
        & Precision & Recall & F1-score & Support \\
        \hline
        0     & 0.91  &    0.96   &   0.94   &   534 \\
        1       & 0.44   &   0.26   &   0.32     &   66  \\
        \hline
        Accuracy      &      &      & 0.88 & 600 \\
        Macro avg     & 0.67 & 0.61 & 0.63 & 600 \\
        Weighted avg  & 0.86 & 0.88 & 0.87 & 600 \\
        \hline
    \end{tabular}
    \caption{Detailed metrics per class}
\end{table}

\begin{figure}[H]
    \centering
    \begin{minipage}[b]{0.48\linewidth}
        \centering
        \includegraphics[width=\linewidth]{images/rf_1.png}
        \label{fig:label1}
    \end{minipage}
    \hspace{0.02\linewidth} % khoảng cách giữa 2 hình
    \begin{minipage}[b]{0.48\linewidth}
        \centering
        \includegraphics[width=\linewidth]{images/rf_2.png}
        \label{fig:label2}
    \end{minipage}
    \caption{Confusion matrix of Random Forest}
    \label{fig:combined}
\end{figure}

\newpage
\subsection{Mô hình XGBoost}
\subsubsection{Giới thiệu}
\hspace{0.5cm} XGBoost (Extreme Gradient Boosting) là một cài đặt tối ưu và mở rộng của thuật toán Gradient Boosting, tập trung vào hiệu năng tính toán và khả năng tổng quát hoá. Mô hình xây dựng dần dần một tập các cây quyết định nông, mỗi cây mới cố gắng sửa lỗi của mô hình hiện tại bằng cách tối thiểu hoá một hàm mất mát thông qua gradient. XGBoost hỗ trợ nhiều kỹ thuật regularization, xử lý giá trị thiếu và song song hoá, nên thường đạt kết quả rất tốt trong các bài toán thi trên Kaggle.
\subsubsection{Kết quả huấn luyện mô hình}
\hspace{0.5cm} Configuration: \{'n\_estimators': 1500, 'learning\_rate': 0.02, 'max\_depth': 15, 'verbosity': 1, 'subsample': 0.85, 'colsample\_bytree': 0.85, 'colsample\_bylevel': 0.9, 'gamma': 0.1, 'reg\_alpha': 0.05, 'reg\_lambda': 1.5, 'min\_child\_weight': 2, 'scale\_pos\_weight': 1\}

\hspace{0.5cm} Kết quả trên tập test:
\begin{itemize}
    \item Accuracy : 0.8817
    \item Precision: 0.8513
    \item Recall   : 0.8817
    \item F1-score : 0.8606
\end{itemize}

\begin{table}[H]
    \centering
    \begin{tabular}{lcccc}
        \hline
        & Precision & Recall & F1-score & Support \\
        \hline
        0     & 0.91   &   0.97   &   0.94   &   534 \\
        1       & 0.41   &   0.18  &    0.25     &   66  \\
        \hline
        Accuracy      &      &      & 0.88 & 600 \\
        Macro avg     & 0.66 & 0.57 & 0.59 & 600 \\
        Weighted avg  & 0.85 & 0.88 & 0.86 & 600 \\
        \hline
    \end{tabular}
    \caption{Detailed metrics per class}
\end{table}

\begin{figure}[H]
    \centering
    \begin{minipage}[b]{0.48\linewidth}
        \centering
        \includegraphics[width=\linewidth]{images/xg_1.png}
        \label{fig:label1}
    \end{minipage}
    \hspace{0.02\linewidth} % khoảng cách giữa 2 hình
    \begin{minipage}[b]{0.48\linewidth}
        \centering
        \includegraphics[width=\linewidth]{images/xg_2.png}
        \label{fig:label2}
    \end{minipage}
    \caption{Confusion matrix of XGBoost}
    \label{fig:combined}
\end{figure}

\newpage
\subsection{Mô hình Gradient Boosting}
\subsubsection{Giới thiệu}
\hspace{0.5cm} Gradient Boosting là một phương pháp ensemble xây dựng mô hình một cách tuần tự, trong đó mỗi mô hình con (thường là cây quyết định nông) được huấn luyện để giảm dần sai số còn lại (residual) của tổ hợp các mô hình trước đó. Mô hình cuối cùng là tổng có trọng số của các mô hình con. Gradient Boosting có khả năng mô tả tốt các quan hệ phi tuyến, nhưng cần điều chỉnh siêu tham số cẩn thận (learning rate, số lượng cây, độ sâu cây, \dots) để tránh overfitting.
\subsubsection{Kết quả huấn luyện mô hình}
\hspace{0.5cm} Configuration: \{'loss': 'log\_loss', 'learning\_rate': 0.02, 'n\_estimators': 1500, 'subsample': 0.85, 'max\_depth': 15, 'min\_samples\_split': 2, 'min\_samples\_leaf': 1, 'max\_features': 'sqrt', 'validation\_fraction': 0.15, 'n\_iter\_no\_change': 30, 'tol': 1e-06\}

\hspace{0.5cm} Kết quả trên tập test:
\begin{itemize}
    \item Accuracy : 0.8900
    \item Precision: 0.8628
    \item Recall   : 0.8900
    \item F1-score : 0.8681
\end{itemize}

\begin{table}[H]
    \centering
    \begin{tabular}{lcccc}
        \hline
        & Precision & Recall & F1-score & Support \\
        \hline
        0     & 0.91  &    0.98   &   0.94   &   534 \\
        1       & 0.50   &   0.20  &    0.28     &   66  \\
        \hline
        Accuracy      &      &      & 0.89 & 600 \\
        Macro avg     & 0.70 & 0.59 & 0.61 & 600 \\
        Weighted avg  & 0.86 & 0.89 & 0.87 & 600 \\
        \hline
    \end{tabular}
    \caption{Detailed metrics per class}
\end{table}

\begin{figure}[H]
    \centering
    \begin{minipage}[b]{0.48\linewidth}
        \centering
        \includegraphics[width=\linewidth]{images/gr_1.png}
        \label{fig:label1}
    \end{minipage}
    \hspace{0.02\linewidth} % khoảng cách giữa 2 hình
    \begin{minipage}[b]{0.48\linewidth}
        \centering
        \includegraphics[width=\linewidth]{images/gr_2.png}
        \label{fig:label2}
    \end{minipage}
    \caption{Confusion matrix of Gradient Boosting}
    \label{fig:combined}
\end{figure}

\newpage
\subsection{Mô hình Naive Bayes}
\subsubsection{Giới thiệu}
\hspace{0.5cm} Naive Bayes là họ mô hình xác suất dựa trên định lý Bayes, với giả định mạnh mẽ rằng các thuộc tính là độc lập với nhau khi đã biết nhãn lớp (giả định ``naive''). Mô hình ước lượng xác suất có điều kiện từ dữ liệu huấn luyện và suy ra xác suất hậu nghiệm cho từng lớp. Dù giả định đơn giản, Naive Bayes thường cho kết quả khá tốt,
đặc biệt trên dữ liệu văn bản hoặc dữ liệu có số chiều cao, và có tốc độ huấn luyện, dự đoán rất nhanh.
\subsubsection{Kết quả huấn luyện mô hình}

\hspace{0.5cm} Kết quả trên tập test:
\begin{itemize}
    \item Accuracy : 0.8583
    \item Precision: 0.8666
    \item Recall   : 0.8583
    \item F1-score : 0.8623
\end{itemize}

\begin{table}[H]
    \centering
    \begin{tabular}{lcccc}
        \hline
        & Precision & Recall & F1-score & Support \\
        \hline
        0     & 0.93   &   0.91  &    0.92   &   534 \\
        1       & 0.37   &   0.42   &   0.40     &   66  \\
        \hline
        Accuracy      &      &      & 0.86 & 600 \\
        Macro avg     & 0.65 & 0.67 & 0.66 & 600 \\
        Weighted avg  & 0.87 & 0.86 & 0.86 & 600 \\
        \hline
    \end{tabular}
    \caption{Detailed metrics per class}
\end{table}

\begin{figure}[H]
    \centering
    \begin{minipage}[b]{0.48\linewidth}
        \centering
        \includegraphics[width=\linewidth]{images/nv_1.png}
        \label{fig:label1}
    \end{minipage}
    \hspace{0.02\linewidth} % khoảng cách giữa 2 hình
    \begin{minipage}[b]{0.48\linewidth}
        \centering
        \includegraphics[width=\linewidth]{images/nv_2.png}
        \label{fig:label2}
    \end{minipage}
    \caption{Confusion matrix of Naive Bayes}
    \label{fig:combined}
\end{figure}

\newpage
\subsection{Mô hình MLP}
\subsubsection{Giới thiệu}
\hspace{0.5cm} Multilayer Perceptron (MLP) là một dạng mạng nơ-ron truyền thẳng (feedforward neural network) gồm một lớp vào, một hoặc nhiều lớp ẩn và một lớp đầu ra. Mỗi lớp gồm nhiều nút (neuron) kết nối đầy đủ với lớp tiếp theo, dùng các hàm kích hoạt phi tuyến (ReLU, sigmoid, \dots) để học các quan hệ phức tạp giữa đầu vào và đầu ra.
MLP có khả năng biểu diễn mạnh, nhưng cần số lượng dữ liệu đủ lớn, kỹ thuật regularization và tối ưu siêu tham số để đạt hiệu quả cao và tránh overfitting.
\subsubsection{Kết quả huấn luyện mô hình}
\hspace{0.5cm} Configuration: \{'hidden\_layer\_sizes': (300, 200, 100), 'activation': 'relu', 'solver': 'adam', 'max\_iter': 5000, 'alpha': 0.0001, 'learning\_rate': 'adaptive', 'early\_stopping': True, 'validation\_fraction': 0.2, 'n\_iter\_no\_change': 30, 'batch\_size': 64\}

\hspace{0.5cm} Kết quả trên tập test:
\begin{itemize}
    \item Accuracy : 0.8417
    \item Precision: 0.8406
    \item Recall   : 0.8417
    \item F1-score : 0.8411
\end{itemize}

\begin{table}[H]
    \centering
    \begin{tabular}{lcccc}
        \hline
        & Precision & Recall & F1-score & Support \\
        \hline
        0     & 0.91   &   0.91   &   0.91   &   534 \\
        1       & 0.28   &   0.27   &   0.27     &   66  \\
        \hline
        Accuracy      &      &      & 0.84 & 600 \\
        Macro avg     & 0.59 & 0.59 & 0.59 & 600 \\
        Weighted avg  & 0.84 & 0.84 & 0.84 & 600 \\
        \hline
    \end{tabular}
    \caption{Detailed metrics per class}
\end{table}

\begin{figure}[H]
    \centering
    \begin{minipage}[b]{0.48\linewidth}
        \centering
        \includegraphics[width=\linewidth]{images/mlp_1.png}
        \label{fig:label1}
    \end{minipage}
    \hspace{0.02\linewidth} % khoảng cách giữa 2 hình
    \begin{minipage}[b]{0.48\linewidth}
        \centering
        \includegraphics[width=\linewidth]{images/mlp_2.png}
        \label{fig:label2}
    \end{minipage}
    \caption{Confusion matrix of Multilayer Perceptron}
    \label{fig:combined}
\end{figure}
\newpage
\section{Đánh giá và so sánh mô hình}
\begin{table}[H]
    \centering
    
    \begin{tabular}{lccccc}
        \hline
        Model           & Accuracy & Precision & Recall  & F1-Score & N\_Features \\
        \hline
        GRADIENTBOOSTING & 0.890000 & 0.862822 & 0.890000 & 0.868073 & 36 \\
        RANDOMFOREST     & 0.881667 & 0.860213 & 0.881667 & 0.867911 & 36 \\
        XGBOOST          & 0.881667 & 0.851349 & 0.881667 & 0.860604 & 36 \\
        NAIVEBAYES       & 0.858333 & 0.866648 & 0.858333 & 0.862253 & 36 \\
        MLP              & 0.841667 & 0.840611 & 0.841667 & 0.841136 & 36 \\
        SVM              & 0.808333 & 0.825987 & 0.808333 & 0.816747 & 36 \\
        DECISIONTREE     & 0.795000 & 0.837561 & 0.795000 & 0.813762 & 36 \\
        KNN              & 0.768333 & 0.844138 & 0.768333 & 0.799046 & 36 \\
        LOGREG           & 0.753333 & 0.849549 & 0.753333 & 0.790335 & 36 \\
        \hline
    \end{tabular}
    \caption{So sánh hiệu quả các mô hình học máy}
\end{table}

\hspace{0.5cm} Dựa trên Bảng trên, có thể thấy nhóm mô hình \textbf{ensemble dựa trên cây} cho hiệu quả tốt nhất. Mô hình \textbf{Gradient Boosting} đạt Accuracy cao nhất (xấp xỉ $0.89$), đồng thời có Precision, Recall và F1-score đều ở mức cao, nên được xem là mô hình toàn diện nhất. Tiếp theo là \textbf{Random Forest} và \textbf{XGBoost} với Accuracy khoảng 0.88 và F1-score đều trên 0.86, cho thấy ba mô hình ensemble này đều học tốt các quan hệ phi tuyến và tương tác phức tạp giữa các đặc trưng.

\hspace{0.5cm} Nhóm mô hình \textbf{Naive Bayes} và \textbf{MLP} cho kết quả khá, với Accuracy trong khoảng $0.84$ đến $0.86$ và F1-score ở mức ổn định. Các mô hình còn lại như \textbf{SVM}, \textbf{Decision Tree}, \textbf{KNN} và \textbf{Logistic Regression} có Accuracy và F1-score thấp hơn,
trong đó Logistic Regression đạt Accuracy thấp nhất. Tuy nhiên, một số mô hình đơn giản như Logistic Regression hay KNN vẫn có Precision tương đối cao, nghĩa là khi đã dự đoán dương tính thì thường đúng, nhưng Recall thấp cho thấy bỏ sót khá nhiều trường hợp khách hàng thực sự đăng ký.

\hspace{0.5cm} Vì tất cả mô hình đều sử dụng cùng một tập 36 đặc trưng, sự khác biệt chủ yếu đến từ khả năng mô hình hóa của từng thuật toán, và kết quả cho thấy các mô hình ensemble là lựa chọn phù hợp hơn cho bài toán này.

\newpage
\section{Kết luận và hướng phát triển}
\subsection{Kết luận}
\hspace{0.5cm} Trong bài tập lớn này, nhóm đã xây dựng một quy trình khai phá dữ liệu hoàn chỉnh cho bài toán dự đoán khả năng khách hàng đăng ký tiền gửi có kỳ hạn dựa trên bộ dữ liệu marketing của ngân hàng. Quy trình bao gồm các bước: khảo sát và tiền xử lý dữ liệu, xây dựng nhiều mô hình học máy khác nhau và cuối cùng là đánh giá, so sánh hiệu quả của các mô hình trên cùng một tập dữ liệu.

\hspace{0.5cm} Từ các kết quả thực nghiệm trình bày ở Phần 3, 4 và 5 có thể rút ra một số nhận xét chính như sau:
\begin{itemize}
    \item Các bước tiền xử lý (làm sạch dữ liệu, mã hoá biến phân loại, chuẩn hoá/chuẩn chỉnh dữ liệu, xử lý dữ liệu mất cân bằng, \dots) có ảnh hưởng rất lớn đến chất lượng dự đoán của mô hình. Việc xử lý đúng giúp mô hình ổn định hơn và cải thiện các chỉ số đánh giá.
    
    \item Những mô hình tuyến tính đơn giản (ví dụ như Logistic Regression) cho kết quả ổn định, dễ huấn luyện và dễ diễn giải, trong khi các mô hình phi tuyến phức tạp hơn (như cây quyết định, rừng ngẫu nhiên, SVM, \dots) có khả năng khai thác mối quan hệ phi tuyến giữa các biến, nhờ đó cải thiện hiệu suất dự đoán trong nhiều trường hợp.
    
    \item Các đặc trưng liên quan đến lịch sử liên hệ và cường độ chiến dịch marketing (thời lượng cuộc gọi, số lần liên hệ, kết quả các chiến dịch trước, \dots) thường mang nhiều thông tin hơn so với một số đặc trưng thông tin cá nhân, qua đó cho thấy tầm quan trọng của việc theo dõi hành vi tương tác của khách hàng.
\end{itemize}
\subsection{Hướng phát triển}

\hspace{0.5cm} Trong phạm vi bài tập lớn, nhóm chỉ mới dừng lại ở một số mô hình cơ bản và cách tiền xử lý dữ liệu tương đối đơn giản. Dựa trên kết quả đạt được, nhóm đề xuất một số hướng phát triển trong tương lai:

\begin{itemize}
    \item Áp dụng các thuật toán mạnh hơn cho bài toán phân loại như LightGBM, CatBoost hoặc các mô hình mạng nơ-ron đơn giản để xem xét
    khả năng cải thiện thêm hiệu suất dự đoán.

    \item Xây dựng mô hình dựa trên dữ liệu theo thời gian thực. 

    \item Tích hợp với các công nghệ mới như Apache Spark để xử lý dữ liệu lớn, Google BigQuery làm Data Warehouse... 

    \item Triển khai mô hình: Xây dựng một API hoặc một giao diện web đơn giản cho phép nhập thông tin khách hàng và trả về xác suất đăng ký tiền gửi, qua đó mô phỏng bước đầu việc đưa mô hình
    vào sử dụng trong môi trường thực tế.
\end{itemize}

\hspace{0.5cm} Thông qua dự án này, nhóm đã tích lũy được nhiều kinh nghiệm thực tiễn trong việc khảo sát, tiền xử lý dữ liệu, xây dựng và đánh giá mô hình học máy. Những kỹ năng này sẽ là nền tảng vững chắc cho các nghiên cứu và dự án sâu hơn trong lĩnh vực trí tuệ nhân tạo và khai thác dữ liệu.
\newpage
%%%%%%%%%%%%%%%%%OUTRO%%%%%%%%%%%%%%%%%%%
\section{Tài liệu tham khảo}

\begin{thebibliography}{9}
\bibitem{tabml}
Data Mining. Concepts and Techniques, 3rd Edition.
\url{https://ia800603.us.archive.org/2/items/datamining_201811/DS-book%20u5.pdf}


\bibitem{tabml}
Marketing Dataset
\url{https://www.kaggle.com/competitions/marketing-data/overview}

\bibitem{tabml}
Introduction to Machine Learning with Python
\url{https://www.nrigroupindia.com/e-book/Introduction%20to%20Machine%20Learning%20with%20Python%20(%20PDFDrive.com%20)-min.pdf}

\bibitem{phamkhanh}
Phạm Đình Khánh.
\textit{Deep AI Book - Random Forest}.
\url{https://phamdinhkhanh.github.io/deepai-book/ch_ml/RandomForest.html}

\bibitem{tabml}
Machine Learning Cơ Bản.
\textit{Random Forest}.
\url{https://machinelearningcoban.com/tabml_book/ch_model/random_forest.html}

\bibitem{geeksforgeeks_xgboost}
GeeksforGeeks.
\textit{XGBoost}.
\url{https://www.geeksforgeeks.org/xgboost/}

\bibitem{viblo_ann}
Viblo.
\textit{Tổng quan về Artificial Neural Network}.
\url{https://viblo.asia/p/tong-quan-ve-artificial-neural-network-1VgZvwYrlAw}

\bibitem{viblo_lr}
Viblo.
\textit{Linear Regression - Hồi quy tuyến tính trong Machine Learning}.
\url{https://viblo.asia/p/linear-regression-hoi-quy-tuyen-tinh-trong-machine-learning-4P856akRlY3}

\bibitem{geeksforgeeks_lr}
GeeksforGeeks.
\textit{ML Linear Regression}.
\url{https://www.geeksforgeeks.org/ml-linear-regression/}

\end{thebibliography}
% \newpage
% \input{outro/phu_luc}
\end{document}
